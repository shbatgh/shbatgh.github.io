%-------------------------
% Resume in Latex
% Author : Jake Gutierrez
% Based off of: https://github.com/sb2nov/resume
% License : MIT
%------------------------

\documentclass[letterpaper,11pt]{article}

\usepackage{latexsym}
\usepackage[empty]{fullpage}
\usepackage{titlesec}
\usepackage{marvosym}
\usepackage[usenames,dvipsnames]{color}
\usepackage{verbatim}
\usepackage{enumitem}
\usepackage[hidelinks]{hyperref}
\usepackage{fancyhdr}
\usepackage[english]{babel}
\usepackage{tabularx}
\input{glyphtounicode}


%----------FONT OPTIONS----------
% sans-serif
% \usepackage[sfdefault]{FiraSans}
% \usepackage[sfdefault]{roboto}
% \usepackage[sfdefault]{noto-sans}
% \usepackage[default]{sourcesanspro}

% serif
% \usepackage{CormorantGaramond}
% \usepackage{charter}


\pagestyle{fancy}
\fancyhf{} % clear all header and footer fields
\fancyfoot{}
\renewcommand{\headrulewidth}{0pt}
\renewcommand{\footrulewidth}{0pt}

% Adjust margins
\addtolength{\oddsidemargin}{-0.55in}
\addtolength{\evensidemargin}{-0.55in}
\addtolength{\textwidth}{1in}
\addtolength{\topmargin}{-.60in}
\addtolength{\textheight}{1.10in}
\linespread{0.96}


\urlstyle{same}

\raggedbottom
\raggedright
\setlength{\tabcolsep}{0in}

% Sections formatting
\titleformat{\section}{
  \vspace{-8.5pt}\scshape\raggedright\large
}{}{0em}{}[\color{black}\titlerule \vspace{-5pt}]

% Ensure that generate pdf is machine readable/ATS parsable
\pdfgentounicode=1

%-------------------------
% Custom commands
\newcommand{\resumeItem}[1]{
  \item\small{
    {#1}
  }\vspace{-2.5pt}
}

\newcommand{\resumeSubheading}[4]{
  \vspace{-3pt}\item
    \begin{tabular*}{0.95\textwidth}[t]{l@{\extracolsep{\fill}}r}
      \textbf{#1} & #2 \\
      \textit{\small#3} & \textit{\small #4} \\
    \end{tabular*}\vspace{-7pt}
}

\newcommand{\resumeSubSubheading}[2]{
    \item
    \begin{tabular*}{0.95\textwidth}{l@{\extracolsep{\fill}}r}
      \textit{\small#1} & \textit{\small #2} \\
    \end{tabular*}\vspace{-7pt}
}

\newcommand{\resumeProjectHeading}[2]{
    \item
    \begin{tabular*}{0.95\textwidth}{l@{\extracolsep{\fill}}r}
      \small#1 & #2 \\
    \end{tabular*}\vspace{-7pt}
}

\newcommand{\resumeSubItem}[1]{\resumeItem{#1}\vspace{-6pt}}

\renewcommand\labelitemii{$\vcenter{\hbox{\tiny$\bullet$}}$}

\newcommand{\resumeSubHeadingListStart}{\begin{itemize}[leftmargin=0.15in, label={}]}
\newcommand{\resumeSubHeadingListEnd}{\end{itemize}}
\newcommand{\resumeItemListStart}{\begin{itemize}}
\newcommand{\resumeItemListEnd}{\end{itemize}\vspace{-5pt}}

%-------------------------------------------
%%%%%%  RESUME STARTS HERE  %%%%%%%%%%%%%%%%%%%%%%%%%%%%


\begin{document}

%----------HEADING----------
\begin{center}
    \textbf{\Huge \scshape Samuel Boccara} \\ \vspace{1pt}
    \small (646) 574-4702 $|$ \href{mailto:shboccara@gmail.com}{\underline{shboccara@gmail.com}} 
\end{center}

\vspace{-15pt}
%-----------EDUCATION-----------
\section{Education}
  \resumeSubHeadingListStart
    \resumeSubheading
      {University of Maryland: College Park}{College Park, MD}
      {Bachelor's of Science in Computer Science \& Mathematics GPA: 4.0/4.0}{Expected May 2028}
      \resumeItemListStart
        \resumeItem{Select Coursework: Multivariable Calc, Linear Algebra, and Differential Equations I \& II, Algorithms, Computer Systems, Programming Languages, Discrete Structures $|$ Activities: BigThink (Audio-ML), Smith Investment Fund}
      \resumeItemListEnd
    \resumeSubheading
      {Hunter College High School}{New York, NY}
        {GPA: 3.9/4.0 $|$ SAT: 1570 $|$ Regeneron ISEF Finalist, ML Club Co-President}{Sep. 2019 -- Jun. 2025}
  \resumeSubHeadingListEnd

%-----------EXPERIENCE-----------
\section{Experience}
  \resumeSubHeadingListStart
    \resumeSubheading
      {Undergraduate Research Assistant}{Oct. 2025 -- Present}
      {University of Maryland Institute for Health Computing}{North Bethesda, MD}
      \resumeItemListStart
        \resumeItem{Researched SOTA single-cell RNA-seq modeling literature (Transformers, Diffusion, VAEs, Optimal Transport) and translated it into end-to-end benchmarks (training + prediction), comparing models against the lab’s in-house model across standardized dataset splits.}
        \resumeItem{Implemented LoRA fine-tuning to improve model performance while reducing compute, iterating on rank/target modules/training schedules.}
        \resumeItem{Stabilized training/inference (formats, shapes, sparse/dense, GPU memory, learning rate) and refactored the model for Hugging Face compatibility (save/load, configs).}
      \resumeItemListEnd
      
    \resumeSubheading
      {Research Intern}{Jun. 2024 -- Aug. 2024}
      {Institut Langevin}{Paris, FR}
      \resumeItemListStart
        \resumeItem{Researched potential applications for generative image models in healthcare, focused on improving denoising and segmentation model performance with synthetic data generation}
        \resumeItem{Built a CNN for denoising + super-resolution of nerve images using fastai and fine-tuned Stable Diffusion 1.5 to generate synthetic corneal nerve images indistinguishable from real data.}
      \resumeItemListEnd
  \resumeSubHeadingListEnd


%-----------PROJECTS-----------
\section{Programming Projects}
    \resumeSubHeadingListStart
      \resumeProjectHeading
          {\textbf{Automated 3D Visualization of Cell Behavior} $|$ \emph{Python, Pytorch, Cellpose}}{}
          \resumeItemListStart
            \resumeItem{Selected as a Regeneron ISEF 2025 Finalist, representing New York City at the international competition.}
            \resumeItem{Designed and implemented an AI pipeline to automatically segment cells in microscopy videos and reconstruct 3D meshes, enabling quantitative analysis of cell dynamics.}
            \resumeItem{Collaborated with Columbia’s Kalderon Lab to deploy the software for germarium dataset analysis, producing 3D visualizations that enabled novel biological findings.}
            \resumeItem{Built an unsupervised clustering framework capable of running spectral, k-means, and HDBSCAN clustering, automatically selecting the best-performing method based on evaluation scores.}
          \resumeItemListEnd
        \resumeProjectHeading
          {\textbf{Arbitrage Trading Bot} $|$ \emph{Python, Websockets, REST, NLP}}{}
          \resumeItemListStart
            \resumeItem{Built a Python-based cross-exchange arbitrage bot for Polymarket + Kalshi, pulling and standardizing ~50,000 markets (top 25k each via REST) into a consistent format to enable reliable cross-platform comparison and pricing.}
            \resumeItem{Implemented an NLP matching pipeline that uses a lightweight embedding model for semantic candidate retrieval and Gemini Flash API verification to validate matches; achieved ~[X]\% precision / ~[Y]\% recall on a labeled sample of [N] market pairs.}
            \resumeItem{Developed a low-latency monitoring + execution loop using websocket streaming for live quotes/order books, computing fee-aware arbitrage edge plus APR and expected profit; delivering ~[A] ms quote-to-alert latency and ~[B] opportunities/day, and semi-automated order placement gated by explicit user confirmation.}
            \resumeItem{Collected Polymarket L2 order book data for the top 2,000 markets over 4 months, building a clean historical dataset for backtesting and microstructure research (over 100M updates).}
          \resumeItemListEnd
        \resumeProjectHeading
          {\textbf{C++ Linear Algebra and Neural Network Library} $|$ \emph{C++}}{}
          \resumeItemListStart
            \resumeItem{Built a C++ linear algebra library (matrix/vector ops, dynamic memory management) and an MLP framework with backprop, activations, and SGD training.}
          \resumeItemListEnd
    \resumeSubHeadingListEnd

%-----------PUBLICATIONS-----------
\section{Publications}
\begin{itemize}[leftmargin=0in, label={}]
  \item \small \textit{4D Imaging of the Germarium Suggests That Follicle Stem Cells and Follicle Cells Self-Organize Around Germline Cysts.} \\
  Amy Reilein, Taj R Chhabra, \textbf{Samuel Boccara}, Shafrir Pervez, Daniel Kalderon. Abstract accepted for \textbf{oral presentation} at the 67th Annual Drosophila Research Conference, March 2026.
         \vspace{-5pt}

  \item \small \textit{Synthetic Procedural Noise and Neural Networks: Enhancing Biomedical Images with Purely Artificial Data} \\
    Viacheslav Mazlin, \textbf{Samuel Boccara}. Accepted for presentation at SPIE Photonics West 2025.
\end{itemize}


%
%-----------PROGRAMMING SKILLS-----------
\section{Awards \& Skills}
 \begin{itemize}[leftmargin=0in, label={}]
    \small{\item{
       \textbf{Awards}{: ISEF Air Force Research Laboratory Award, USACO Gold} \\
       {\textbf{Skills}: Python, C++, Pytorch, JAX, Numpy, Git}
    }}
 \end{itemize}


%-------------------------------------------
\end{document}